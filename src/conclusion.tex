\section{Вывод}
Различные способы физического кодирования имеют свои достоинства и недостатки.
В целом метод, имеющий преимущества в виде компенсации ошибок или самосинхронизации,
оказывается дороже более простых методов, т.к. требует большее количество уровней
сигнала либо большую полосу пропускания.

Логическое кодирование 4B/5B позволяет гарантированно иметь не более 8 идущих подряд одинаковых уровней сигнала,
а также добавляет механизм обнаружения ошибок,
в то время как скремблирование позволяет не менять размер сообщения, но не гарантирует выигрыш по полосе пропускания.
Также стоит помнить, что информация о методе скремблирования тоже должна быть как-то передана,
поэтому не имеет смысла подбирать идеальный скремблер для каждого сообщения --- это приведет к передаче
всего сообщения в виде записи о методе скремблирования и значительно увеличит время передачи.

Самыми дешевыми методами для передачи сообщения оказались NRZ и NRZI,
поскольку они используют только 2 уровня сигнала и наименьшую полосу пропускания.
В реальных условиях метод нужно выбирать, исходя из известной информации о среде,
приемнике и передатчике.

\section{Список использованной литературы}
\begin{enumerate}
    \item КОМПЬЮТЕРНЫЕ СЕТИ И ТЕЛЕКОММУНИКАЦИИ: ЗАДАНИЯ И ТЕСТЫ --- Т.И. Алиев, В.В. Соснин, Д.Н. Шинкарук
\end{enumerate}
