\section{Порядок выполнения работы}
\begin{enumerate}
    \item \label{stage1} Ознакомиться с постановкой задачи и изучить необходимые теоретические сведения.
    \item \label{stage2} Сформировать исходное сообщение в соответствии с этапом~\ref{stage1}.
    \item \label{stage3} Выполнить физическое кодирование исходного сообщения не менее, чем тремя способами, включая,
    в качестве обязательного, манчестерское кодирование.
    Рассчитать частотные характеристики передаваемого сигнала для рассматриваемых способов кодирования и определить
    требуемую для эффективной передачи сообщения пропускную способность канала связи (этап~\ref{stage2}).
    \item \label{stage4} Выполнить логическое кодирование исходного сообщения, используя избыточное кодирование 4В/5В и скремблирование.
    Рассчитать частотные характеристики передаваемого сигнала для рассматриваемых способов кодирования и определить
    требуемую для эффективной передачи сообщения пропускную способность канала связи (этапы~\ref{stage3} и~\ref{stage4}).
    \item \label{stage5} Выполнить сравнительный анализ рассмотренных способов кодирования и выбрать наилучший способ для передачи исходного
    сообщения (этап~\ref{stage5}).
    \item \label{stage6} Оформить отчёт и сдать его на проверку.
    \item \label{stage7} В назначенное преподавателем время защитить задание.
\end{enumerate}
